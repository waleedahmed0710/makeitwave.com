%% Generated by Sphinx.
\def\sphinxdocclass{report}
\documentclass[letterpaper,10pt,openany,oneside,english]{sphinxmanual}
\ifdefined\pdfpxdimen
   \let\sphinxpxdimen\pdfpxdimen\else\newdimen\sphinxpxdimen
\fi \sphinxpxdimen=.75bp\relax

\PassOptionsToPackage{warn}{textcomp}
\usepackage[utf8]{inputenc}
\ifdefined\DeclareUnicodeCharacter
 \ifdefined\DeclareUnicodeCharacterAsOptional
  \DeclareUnicodeCharacter{"00A0}{\nobreakspace}
  \DeclareUnicodeCharacter{"2500}{\sphinxunichar{2500}}
  \DeclareUnicodeCharacter{"2502}{\sphinxunichar{2502}}
  \DeclareUnicodeCharacter{"2514}{\sphinxunichar{2514}}
  \DeclareUnicodeCharacter{"251C}{\sphinxunichar{251C}}
  \DeclareUnicodeCharacter{"2572}{\textbackslash}
 \else
  \DeclareUnicodeCharacter{00A0}{\nobreakspace}
  \DeclareUnicodeCharacter{2500}{\sphinxunichar{2500}}
  \DeclareUnicodeCharacter{2502}{\sphinxunichar{2502}}
  \DeclareUnicodeCharacter{2514}{\sphinxunichar{2514}}
  \DeclareUnicodeCharacter{251C}{\sphinxunichar{251C}}
  \DeclareUnicodeCharacter{2572}{\textbackslash}
 \fi
\fi
\usepackage{cmap}
\usepackage[T1]{fontenc}
\usepackage{amsmath,amssymb,amstext}
\usepackage[english]{babel}
\usepackage{times}
\usepackage[Bjarne]{fncychap}
\usepackage{sphinx}

\usepackage{geometry}

% Include hyperref last.
\usepackage{hyperref}
% Fix anchor placement for figures with captions.
\usepackage{hypcap}% it must be loaded after hyperref.
% Set up styles of URL: it should be placed after hyperref.
\urlstyle{same}

\addto\captionsenglish{\renewcommand{\figurename}{Fig.}}
\addto\captionsenglish{\renewcommand{\tablename}{Table}}
\addto\captionsenglish{\renewcommand{\literalblockname}{Listing}}

\addto\captionsenglish{\renewcommand{\literalblockcontinuedname}{continued from previous page}}
\addto\captionsenglish{\renewcommand{\literalblockcontinuesname}{continues on next page}}

\addto\extrasenglish{\def\pageautorefname{page}}

\setcounter{tocdepth}{1}



\title{WaveOS™ Operators Guide}
\date{Jan 19, 2019}
\release{0.5.0rc1}
\author{Author(s): Siôn H. Buckler, Wave}
\newcommand{\sphinxlogo}{\vbox{}}
\renewcommand{\releasename}{Release}
\makeindex

\begin{document}

\maketitle
\sphinxtableofcontents
\phantomsection\label{\detokenize{index::doc}}



\chapter{Operators Guide}
\label{\detokenize{index:operators-guide}}
Welcome to the WaveOS™ Operators Guide. Contained in this document is clear and helpful information to assist you in your understanding, use and enjoyment of WaveOS™.

Index:


\section{Changelog}
\label{\detokenize{changelog:changelog}}\label{\detokenize{changelog::doc}}\begin{itemize}
\item {} 
{\color{red}\bfseries{}:release:{}`1.0.1 \textless{}2018-11-30\textgreater{}{}`}

\item {} 
{\color{red}\bfseries{}:feature:{}`01{}`} Just Published using Sphinx.

\end{itemize}


\section{Release Notes and Notices}
\label{\detokenize{releasenotes:release-notes-and-notices}}\label{\detokenize{releasenotes::doc}}
This section provides information about what is new or changed, including urgent issues, Software \& documentation updates, maintenance and new releases.
\begin{itemize}
\item {} 
‘Updates’ are the term used to describe significant changes to our public source code

\item {} 
Twice daily WaveOS™ will check the source code for scripts which can impliment these changes

\item {} 
If a change script is flagged as ready, WaveOS™ will retrieve and run it locally

\item {} 
The last digit(s) of the versions ID will increase to reflect the new update e.g. \sphinxcode{\sphinxupquote{0.0.X}}

\item {} 
Auto-Updates are normally performed every two months

\item {} 
A reboot and downtime of between 30 seconds and 2 minutes is expected

\end{itemize}


\subsection{Version 0.5.0 (Alpha)}
\label{\detokenize{releasenotes:version-0-5-0-alpha}}
In this version release our developement team have removed a barrier which requested new users adjust their router’s subnet to match Wave’s pre-set static IP, before booting up the device for the first time. An issue preventing the Edition from running and subsequently the initial installation from completing was also rectified.
\begin{itemize}
\item {} 
25th Oct 2018: Development of version 0.5.0 begins

\item {} 
The default IP during initial boot was static. The device will now await to be dynamically assigned and IP by the connected router

\item {} 
The dynamically assigned IP (\& Gateway \& DNS) will autonomously convert to a static IP when a ping to 8.8.8.8 succeeds

\item {} 
To avoid conflicting IP addresses on the network, this process will repeat in the even the connection is lost.

\item {} 
1st December 2018: Release for public download

\end{itemize}


\subsubsection{Older Versions}
\label{\detokenize{releasenotes:older-versions}}
Below are references to older version releases and release notes:


\begin{savenotes}\sphinxattablestart
\centering
\sphinxcapstartof{table}
\sphinxcaption{Table 1.0 - archived versions of WaveOS™}\label{\detokenize{releasenotes:id1}}
\sphinxaftercaption
\begin{tabular}[t]{|\X{20}{100}|\X{20}{100}|\X{20}{100}|\X{20}{100}|\X{20}{100}|}
\hline
\sphinxstyletheadfamily 
date
&\sphinxstyletheadfamily 
version
&\sphinxstyletheadfamily 
size
&\sphinxstyletheadfamily 
description
&\sphinxstyletheadfamily 
download link
\\
\hline
01-06-2018
&
0.4.4
&
150Mb
&
improved autonomous install
&
Download 0.4.0 (updates itself to 0.4.4)
\\
\hline
01-04-2018
&
0.4.3
&
150Mb
&
improved autonomous install
&
Download 0.4.0 (updates itself to 0.4.3)
\\
\hline
01-02-2018
&
0.4.2
&
150Mb
&
improved autonomous install
&
Download 0.4.0 (updates itself to 0.4.2)
\\
\hline
01-12-2017
&
0.4.1
&
150Mb
&
improved autonomous install
&
Download 0.4.0 (updates itself to 0.4.1)
\\
\hline
01-10-2017
&
0.4.0
&
150Mb
&
correctly built image (7zip)
&
\sphinxurl{https://mega.nz/\#!4aYSiJiS!S2VeWes\_0SPgtxJD2yVxAYrVlQEsvFT\_D1ft0Tt5As8}
\\
\hline
01-08-2017
&
0.3.0
&
500Mb
&
attempt 1 to build it right
&
Not Public
\\
\hline
01-06-2017
&
0.2.0
&
500Mb
&
compressed copy of image
&
\sphinxurl{https://mega.nz/\#!YbpRgIKS!GEpuU9cKBb2Ef0SaEXsjgkXiZDcnIwBwt7lH-fQRA-A}
\\
\hline
01-04-2017
&
0.1.0
&
2Gb
&
proof-of-concept (10.0.0.x)
&
\sphinxurl{https://mega.nz/\#!8HwW1JKZ!Hs1HQCEKzNIabAGGtAZ5t2O9ppSve9canZFbBAcl8v8}
\\
\hline
\end{tabular}
\par
\sphinxattableend\end{savenotes}


\subsubsection{Version 0.4.4}
\label{\detokenize{releasenotes:version-0-4-4}}
Published in Winter 2018. 500Mb .ISO (150Mb compressed). Lightweight in size \& quick to download. Installs and configures autonomously in around 10 minutes during first boot. Now contains a remote updating feature. Some new {\hyperref[\detokenize{releasenotes:known-and-corrected-issues}]{\emph{issues}}}, still need resolving by next version


\subsubsection{Version 0.4.0}
\label{\detokenize{releasenotes:version-0-4-0}}
Published in Summer 2018. The image is much lighter (150Mb). But there were serious faults with this release and it should be disregarded. Possibly due to how we were compiling the image. We introduced PiShrink and brought in someone who was familiar at image compression before releasing another version. A point worth noting is that we are installing DietPi to an SD Card, alter a config file then taking a local copy of the image and compressing it. Which is an around the houses way to go about it we need to address. Perhaps we can fork DietPi, alter the config to use ours then compile it ourself rather than reverse engineering their final build.


\subsubsection{Version 0.3.0}
\label{\detokenize{releasenotes:version-0-3-0}}
Published in Spring 2018 as proof-of-concept, demonstrating how easily our solution could be downloaded from our website (for free), copied to a Micro SD Card and upon inserSiôn (into any of the 19 million Single Board Computers in circulation) the device and the software operating it would perform as intended, without any programming knowledge or configuration required e.g. completely ‘plug \& play’. This demonstrated the methodology of quick deployment and scaling internationally. WaveOS™ version 0.1.0 also demostrated how product assembly could occur with a non-skilled/ robotic workforce. Faults with this version release include download time (it’s 2GB) and restriction to the exact device type the source ran on. Since it’s a snapshot (copy) only, outdates software is actually being transfered, instead of the latest source code being obtained during first boot. There are also many features not included in this image.


\subsection{Known and Corrected Issues}
\label{\detokenize{releasenotes:known-and-corrected-issues}}\begin{description}
\item[{Below is a table of pending issues which have been reported to our team.}] \leavevmode
These issues will be cleared from this list as and when they are remedied.

\end{description}


\begin{savenotes}\sphinxattablestart
\centering
\sphinxcapstartof{table}
\sphinxcaption{Table 1.1 - Known Issues}\label{\detokenize{releasenotes:id2}}
\sphinxaftercaption
\begin{tabular}[t]{|\X{10}{100}|\X{10}{100}|\X{20}{100}|\X{60}{100}|}
\hline
\sphinxstyletheadfamily 
date
&\sphinxstyletheadfamily 
subject
&\sphinxstyletheadfamily 
version
&\sphinxstyletheadfamily 
description
\\
\hline
01-08-2018
&
static ip
&
0.4.4
&
Cannot be expected to change my router settings to suite Wave.
\\
\hline
01-08-2018
&
cgi
&
0.4.4
&
Buttons not working on Edition Select stage. Can’t select Edition to complete installation of WaveOS®
\\
\hline
\end{tabular}
\par
\sphinxattableend\end{savenotes}

\sphinxstylestrong{Comments} - If connection is lost/ re-established, version 0.5+ will stop services, return to dynamic then change to static (avoiding network conflics). The script required to convert a dynamic address to a static one, after the ethernet is inserted or removed and re-inserted ( \sphinxcode{\sphinxupquote{ifup}} / \sphinxcode{\sphinxupquote{ifdown}} ) must be on the sd card image itself and obviously not a remote location, since there will be no outside connection until said script establishes it

\sphinxstylestrong{Comments} - This idea of converting a dynamic IP to a static one should be done in DietPi (DietPi bridges the gap between the Linux Operating System and Wave® Software layer). Their team is also aware of the suggestion since we submitted it via their GitHub repository in early 2018. However Wave® aims to phase out DietPi from its final solution and build itself upon Linux directly. DietPi has been used for bootstrapping and startup only, since it automates many of the functions Wave® depends on and the DietPi team have resolved many of the issues integrating the various 3rd party applications e.g. Emby, Netstats, PiHole etc.


\subsection{Recently Updated Topics}
\label{\detokenize{releasenotes:recently-updated-topics}}
Nothing significant to report


\section{Preperation \& Planning}
\label{\detokenize{preperation:preperation-planning}}\label{\detokenize{preperation::doc}}
This guide has been written to help you prepaire yourself and your home for Wave.
There is very little preperation and planning required with Wave, since it is intended to be a plug \& play solution.
This guide also presumes you have no technical knowledge and little knowledge of what to expect from our software and products, how the technology works or how to use it.


\subsection{Expectations}
\label{\detokenize{preperation:expectations}}
\sphinxstylestrong{Obtaining Wave:} - WaveOS™ is a free software which operates devices called Single Board Computers. More specifically the type know as: Raspberry Pi 3. WaveOS™ can be downloaded from our website for free. So if you have a Raspberry Pi already you can use and enjoy our technology without risking much.

If you don’t already have a device, they can be purched from the official \sphinxhref{https://makeitwave.com}{Wave® website}. The checkout accepts payment through Paypal. Your order will be immidiately passed to a UK distributor who will ship the product to you. The manufacturer of this hardware device is Sony and the delivery will include all cables and connectors needed to use and enjoy the product to its full potential.

\sphinxstylestrong{Setting Up Wave:} - After unboxing the device and ensuring the Micro SD Card is inserted(containing the latest version of WaveOS™), you must connect the device to your internet router using the Ethernet Cable provided and connect the Power Cable to the power outlet. You should also connect the HDMI Cable to a display if you wish to oversea the boot-up sequence.

It takes around 10 minutes for the device to boot up for the first time, so please be patience. Upon successful completion the device will be broadcasing the Wi-Fi SSID: wave-hotspot. Connect to this from any Wireless Device (phone, laptop etc), then open your web browser and type \sphinxcode{\sphinxupquote{192.168.42.1}} or \sphinxcode{\sphinxupquote{https://wave/}} in the address bar.

\sphinxstylestrong{Our Hardware Solutions …} a solution which will reduce your energy and internet bills each month. However it is important to understand that this is an end goal we are working towards which requires users of the technology to participate by procuring the technology, leaving it connected to internet and power.

\sphinxstylestrong{Our Team …}

\sphinxstylestrong{Our Company …} accountability. which we have to partners, shareholders, distributors, staff, inland revenue, postal service etc.

\sphinxstylestrong{Your Energy Company …}

\sphinxstylestrong{Your Internet Service Provider …}

Our charter to make energy \& internet free is only a milestone of a bigger plan to make energy and internet an inalienable birth right (rather than an affordable privelage) for our next generation, so you need not tell us to hop to it, since we are already very motivated to achieve this both technologically, politically and in our capacity as humanitarians. So allow yourself a moment to come away from all the hype and marketing material which has brought you to this paragraph and lets give our saner heads a chance to prevail once again. We don’t want to have a single misunderstanding with you, since it’s us and you entering into a relationship here, so we take this opportuniy to manage your expectations of Wave® as we trust you will yourself.

We presume you’re smart, we consider ourselves smart too, so what we’d like at this first stage is to make sure you’re perfectly clear of what you’re getting with Wave. And hopeful give you a basic idea of how it all works if you don’t know already. We’d hate to think you’re running around worrying your friends and family into believing you’ve just gone mad or joined a cult. Equally we don’t want anyone claiming we’re riding off into the sunset with everyone’s hard earnt money, while you end up with a plastic box which does nothing for you. It’s better you do your homework now, establish some facts, then know for sure exactly what you’re getting yourself into with Wave, specifically since our technology may soon be entrusted to manage your internet, energy, entertainment, home and even your vehicles location depending on your level of trust in our technology.

The technology is still early stage, we won’t lie to you (it’s 2018 at the time of writing this). And technology can take months and even years to hit maturity. So please don’t hold your quality expectation as high as Facebook or Google’s products and service, which are supported with billions in revenue and hundreds of thousands of employees. The team behind Wave® is tiny and we’re a small ambitious startup. So this is ultimately a relationship between you and us, which could go on for many years if all goes well. We have had some great companies support us over these last few years and we sincerly seek the same long and prosperous relationship with our technologies users. Over the next few months and years we hope you’ll visualise our team working night and day to lower your utility bills every time you look at a utility bill or the device(s) around your home. Remembering that you bought the thing a while back and it’s not let you down since.

Sony Manufacture the Hardware in Pencoed in Wales, the physical technology is called a Single Board Computer, more specifically they’re Raspberry Pi’s. At the time of writing this there are 19million Single Board Computers in circulation and Wave’s is being developed for all kinds of these devices. But at this stage we are especially focused on the Raspberry Pi 3 until our user base reaches critical mass. So if you have one already then great, download WaveOS™ for free from \sphinxhref{https://makeitwave.com}{our website}, and you’re good to go. If not, purchase a device through \sphinxhref{https://makeitwave.com}{our website}, and we’ll get one sent out to you immidiately.

The free internet is achieved with a number of pioneering technologies. We currently use JSE Coin to mine cryptocurrency from reduntant processing power of devices you connect to Wave® to interface with the solution. We also use a technology called PiHole to perform ad-reinserSiôn, which getnerates advertising revenue. Ultimately you should not notice these two technologies, since you likely had no idea what adnetwork was serving ads to the websites you visited beforehand, and you likely had no knowledge of the redundant processing power of your personal devices e.g. laptop, phone, tablet, tv etc. In any case these funds autonomously collate into a single digital currency wallet corresponding to your specific devices unique processor ID. Then we use a technology called Bit-refill and Cryptocurrency Smart Contracts to autonomously transfer that balance to your Internet Service Provider each month. WaveOS™ uses your Public IP to determine who your buildings ISP is (not who you are) in order to automate payments to them each month. We call this whole set of intelligent systems: Surf-on-Wave.

The free energy is also achieved with several technologies but the methodology and whole approach is very difference. Your energy bill isn’t paid for in any way, Wave® actually just helps you lower consumption. We use OpenHaB in our software, so that you have full control of your homes wireless appliances, lights, plug sockets etc. And we use a technology called Dataplicity so that you can access these controls from anywhere. Wave® interfaces with a technology called Sense, which uses AI and crowdsourcing to determine what’s active in your home and what energy these devices consume. Optional extras may include Sensibo, which converts your AirConditionings Infrared Eye into Wi-Fi so it can work with Wave. We also grant access to multiple users, so your households co-habbitors can all monitor and control energy. Furthermore you can easily create logical rules, automation and it is this, all together, which saves your energy consumption and subsequently reduces your bills each month. If you have solar or wind and/or batteries, Wave® can also join the mix to improve production of enery.  We call this whole set of intelligent systems: Juice-on-Wave.

These little Single Board Computers are really very powerful, but WaveOS™ does test them to their limits. Never before has so many systems been run simultaniously on one device. Wave® really is like trying to run the pentagon from your watch. Or more ambitiously, upgrade everyones watches so we all have insane amounts of power at our finger tips. It was only a few years ago all of the technologies contained in Wave® was actually held on equitment which had to be liquid cooled and filled a 19” rack mountable cabinate, which has it’s own small room within a building. If Wave® were more intelligent it would make Shakespear look like it was written in crayon. And it would read it to you.

As a small disclaimer you should also be aware that none of what is being said here pertains to one specific product, but instead \sphinxhref{https://makeitwave.com}{our product line}, as a whole. You can achieve much of what is being said with just the Wave® Home Hub™, which is a Single Board Computer running the \sphinxstylestrong{Home Hub Edition™} of WaveOS™ (Free to Download). But you may need a solution for monitoring the energy and extending its signal so that the Energy Monitor™ can communicate with the Home Hub™. In this case multiple products and Software Editions will need to be in operation within your home - not just one plug \& play device.


\subsection{Share Our Experience}
\label{\detokenize{preperation:share-our-experience}}
As of 2018 and version 0.4.4 of WaveOS™ there is no issues obtaining a Single Board Computer, copying the Software to the Micro SD Card and inserting it into the device before powering it up:
\begin{itemize}
\item {} 
The first thing to expect is the device to become a Wi-Fi hotspot a few minutes after boot-up. You can then connect to it in order to view the main menu and select the Edition of the Software you’d like installed.

\item {} 
At this point everyone beta testing the technology is waiting for update 0.4.5 which will permit the first Edition to be installed. It is a one-click installation however we have disabled this for the moment to prevent anyone going further. We move forward together on this as a unit.

\item {} 
We aim to release version 0.4.5 and get going with this in December 2018. 0.4.5 is an upgrade, not an update. Which means the software will need to be re-downloaded and re-written to the Micro SD Card. But this is expected to be the last time you will need to do this for our convenience. The remainder of the upgrades will be done automatically over the internet. We already have this ability developed which you will notice during the boot up of version 0.4 if you have it hooked up to a monitor. It updates from 0.4.1 to 0.4.2 to 0.4.3 then 0.4.4.

\item {} 
Before 2019 the only Edition of WaveOS™ you will be able to select and install will be the \sphinxstylestrong{Home Hub Edition™}. But we are accepting orders for the various devices with the various editions installed.

\item {} 
Several of the applications for the \sphinxstylestrong{Home Hub Edition™} install autonomously and work great. Specifically the Media Center (Emby), IP Camera DVR (MotionEye) and Smart Home Control Application (OpenHaB). The vehicle tracking application and Energy Monitoring is still receiving our attention, as are the Vehicle Tracking and Energy Monitoring Editions, which power the devices which make these two solutions possible.

\item {} 
We have a Facebook Page where we welcome public discussions around this technology. We also welcome emails if you want a more private conversation or have any questions, queries or concerns. Our email address is \sphinxhref{mailto:info@makeitwave.com}{info@makeitwave.com}

\end{itemize}


\section{Getting Started Guide}
\label{\detokenize{gettingstarted:getting-started-guide}}\label{\detokenize{gettingstarted::doc}}

\subsection{hyperlink}
\label{\detokenize{gettingstarted:hyperlink}}
Hyperlink \sphinxhref{http://ArchLinuxarm.org/platforms/armv6/raspberry-pi}{here},


\subsection{command lines}
\label{\detokenize{gettingstarted:command-lines}}
like this: \sphinxcode{\sphinxupquote{192.168.0.x}}, \sphinxcode{\sphinxupquote{10.0.0.14x}} or

Enough of networking for now. We’ll set a proper network configuration later in this guide, but first some \sphinxstyleemphasis{musthaves}.


\subsubsection{text block}
\label{\detokenize{gettingstarted:text-block}}
\fvset{hllines={, ,}}%
\begin{sphinxVerbatim}[commandchars=\\\{\}]
\PYG{n}{passwd}  \PYG{c+c1}{\PYGZsh{} change root password to something important}
\PYG{n}{rm} \PYG{o}{\PYGZhy{}}\PYG{n}{rf} \PYG{o}{/}\PYG{n}{etc}\PYG{o}{/}\PYG{n}{localtime}  \PYG{c+c1}{\PYGZsh{} dont care about this}
\PYG{n}{ln} \PYG{o}{\PYGZhy{}}\PYG{n}{s} \PYG{o}{/}\PYG{n}{usr}\PYG{o}{/}\PYG{n}{share}\PYG{o}{/}\PYG{n}{zoneinfo}\PYG{o}{/}\PYG{n}{Europe}\PYG{o}{/}\PYG{n}{Prague} \PYG{o}{/}\PYG{n}{etc}\PYG{o}{/}\PYG{n}{localtime}  \PYG{c+c1}{\PYGZsh{} set appropriate timezone}
\PYG{n}{echo} \PYG{l+s+s2}{\PYGZdq{}}\PYG{l+s+s2}{my\PYGZus{}raspberry}\PYG{l+s+s2}{\PYGZdq{}} \PYG{o}{\PYGZgt{}}  \PYG{o}{/}\PYG{n}{etc}\PYG{o}{/}\PYG{n}{hostname}  \PYG{c+c1}{\PYGZsh{} set name of your RPi}

\PYG{n}{useradd} \PYG{o}{\PYGZhy{}}\PYG{n}{m} \PYG{o}{\PYGZhy{}}\PYG{n}{aG} \PYG{n}{wheel} \PYG{o}{\PYGZhy{}}\PYG{n}{s} \PYG{o}{/}\PYG{n}{usr}\PYG{o}{/}\PYG{n+nb}{bin}\PYG{o}{/}\PYG{n}{bash} \PYG{n}{common\PYGZus{}user} \PYG{c+c1}{\PYGZsh{}}
\PYG{n}{groupadd} \PYG{n}{webdata}  \PYG{c+c1}{\PYGZsh{} for sharing}
\PYG{n}{useradd} \PYG{o}{\PYGZhy{}}\PYG{n}{M} \PYG{o}{\PYGZhy{}}\PYG{n}{aG} \PYG{n}{webdata} \PYG{o}{\PYGZhy{}}\PYG{n}{s} \PYG{o}{/}\PYG{n}{usr}\PYG{o}{/}\PYG{n+nb}{bin}\PYG{o}{/}\PYG{n}{false} \PYG{n}{nginx}
\PYG{n}{usermod} \PYG{o}{\PYGZhy{}}\PYG{n}{aG} \PYG{n}{webdata} \PYG{n}{common\PYGZus{}user}

\PYG{n}{visudo}  \PYG{c+c1}{\PYGZsh{} uncomment this line:  \PYGZpc{}wheel ALL=(ALL) ALL}

\PYG{n}{pacman} \PYG{o}{\PYGZhy{}}\PYG{n}{Syu}
\end{sphinxVerbatim}

\sphinxstylestrong{bold text}
\begin{itemize}
\item {} 
bullet

\item {} 
point

\end{itemize}


\begin{savenotes}\sphinxattablestart
\centering
\sphinxcapstartof{table}
\sphinxcaption{Title}\label{\detokenize{gettingstarted:id1}}
\sphinxaftercaption
\begin{tabular}[t]{|\X{25}{100}|\X{25}{100}|\X{50}{100}|}
\hline
\sphinxstyletheadfamily 
Heading row 1, column 1
&\sphinxstyletheadfamily 
Heading row 1, column 2
&\sphinxstyletheadfamily 
Heading row 1, column 3
\\
\hline
Row 1, column 1
&&
Row 1, column 3
\\
\hline
Row 2, column 1
&
Row 2, column 2
&
Row 2, column 3
\\
\hline
\end{tabular}
\par
\sphinxattableend\end{savenotes}


\section{Installation Guide}
\label{\detokenize{installation:installation-guide}}\label{\detokenize{installation::doc}}

\subsection{hyperlink}
\label{\detokenize{installation:hyperlink}}
Hyperlink \sphinxhref{http://ArchLinuxarm.org/platforms/armv6/raspberry-pi}{here},


\subsection{command lines}
\label{\detokenize{installation:command-lines}}
like this: \sphinxcode{\sphinxupquote{192.168.0.x}}, \sphinxcode{\sphinxupquote{10.0.0.14x}} or

Enough of networking for now. We’ll set a proper network configuration later in this guide, but first some \sphinxstyleemphasis{musthaves}.


\subsubsection{text block}
\label{\detokenize{installation:text-block}}
\fvset{hllines={, ,}}%
\begin{sphinxVerbatim}[commandchars=\\\{\}]
\PYG{n}{passwd}  \PYG{c+c1}{\PYGZsh{} change root password to something important}
\PYG{n}{rm} \PYG{o}{\PYGZhy{}}\PYG{n}{rf} \PYG{o}{/}\PYG{n}{etc}\PYG{o}{/}\PYG{n}{localtime}  \PYG{c+c1}{\PYGZsh{} dont care about this}
\PYG{n}{ln} \PYG{o}{\PYGZhy{}}\PYG{n}{s} \PYG{o}{/}\PYG{n}{usr}\PYG{o}{/}\PYG{n}{share}\PYG{o}{/}\PYG{n}{zoneinfo}\PYG{o}{/}\PYG{n}{Europe}\PYG{o}{/}\PYG{n}{Prague} \PYG{o}{/}\PYG{n}{etc}\PYG{o}{/}\PYG{n}{localtime}  \PYG{c+c1}{\PYGZsh{} set appropriate timezone}
\PYG{n}{echo} \PYG{l+s+s2}{\PYGZdq{}}\PYG{l+s+s2}{my\PYGZus{}raspberry}\PYG{l+s+s2}{\PYGZdq{}} \PYG{o}{\PYGZgt{}}  \PYG{o}{/}\PYG{n}{etc}\PYG{o}{/}\PYG{n}{hostname}  \PYG{c+c1}{\PYGZsh{} set name of your RPi}

\PYG{n}{useradd} \PYG{o}{\PYGZhy{}}\PYG{n}{m} \PYG{o}{\PYGZhy{}}\PYG{n}{aG} \PYG{n}{wheel} \PYG{o}{\PYGZhy{}}\PYG{n}{s} \PYG{o}{/}\PYG{n}{usr}\PYG{o}{/}\PYG{n+nb}{bin}\PYG{o}{/}\PYG{n}{bash} \PYG{n}{common\PYGZus{}user} \PYG{c+c1}{\PYGZsh{}}
\PYG{n}{groupadd} \PYG{n}{webdata}  \PYG{c+c1}{\PYGZsh{} for sharing}
\PYG{n}{useradd} \PYG{o}{\PYGZhy{}}\PYG{n}{M} \PYG{o}{\PYGZhy{}}\PYG{n}{aG} \PYG{n}{webdata} \PYG{o}{\PYGZhy{}}\PYG{n}{s} \PYG{o}{/}\PYG{n}{usr}\PYG{o}{/}\PYG{n+nb}{bin}\PYG{o}{/}\PYG{n}{false} \PYG{n}{nginx}
\PYG{n}{usermod} \PYG{o}{\PYGZhy{}}\PYG{n}{aG} \PYG{n}{webdata} \PYG{n}{common\PYGZus{}user}

\PYG{n}{visudo}  \PYG{c+c1}{\PYGZsh{} uncomment this line:  \PYGZpc{}wheel ALL=(ALL) ALL}

\PYG{n}{pacman} \PYG{o}{\PYGZhy{}}\PYG{n}{Syu}
\end{sphinxVerbatim}

\sphinxstylestrong{bold text}
\begin{itemize}
\item {} 
bullet

\item {} 
point

\end{itemize}


\section{General Usage}
\label{\detokenize{end-user:general-usage}}\label{\detokenize{end-user::doc}}

\subsection{What is WaveOS™?}
\label{\detokenize{end-user:what-is-waveos}}
The Wave® Operating System (WaveOS™) is a Free \& Open-Source Linux-Based Software, designed to make any Single Board Computer a Plug \& Play \sphinxstylestrong{Smart Home Hub™} with the unique capability of making internet and energy Freer and/or completely free for the household in which it operates. The solution also features some great open source and free apps for the end-users e.g. IPTV/ Media Center, IPCCTV DVR, IoT Smart Home Control, Vehicle Tracking and Energy Monitoring - All autonomously installed and configured during initial installation, which is also autonomous. Wave® simply copies onto a Micro SD card, inserts into any Single Board Computer connected to the internet and within minutes can be used, enjoyed and benefitted from.


\subsection{Disclaimer}
\label{\detokenize{end-user:disclaimer}}
Keep in mind that although I am a professional engineer with extensive background experience and education, this is the first product of this magnitude I have attempted to develop, the work is ongoing and as of 2018 I am still in the early phases and some time away from a final solution. There is much I have yet to learnt about best practices for Documenting, Open Source Code \& version Control and Systems Integration of the various technologies included in this new Operating System. There are industry standard methodologies this new Operating System is yet to adhear to - I remain humble and open to suggestions at all levels.

Everything you find here at this stage is without waranty and I accept not responsible for any inconveniences or issues that might occur as a result of use of this new Operating System: WaveOS™. As time goes on I can only assure you that less Single Board Computers and Memory Sticks become damaged and/or ‘bricked’ by WaveOS™. Fortunately both are low cost hardware and WaveOS™ is free, so it’s feasible fun and promising technology right now, to say the least.


\subsection{Languages}
\label{\detokenize{end-user:languages}}
The primary language of WaveOS™ will be English. Secondary languages will be introduced using translators which will use the English literature as the primary source of information.


\section{Device Upkeep}
\label{\detokenize{productmaint:device-upkeep}}\label{\detokenize{productmaint::doc}}

\subsection{hyperlink}
\label{\detokenize{productmaint:hyperlink}}
Hyperlink \sphinxhref{http://ArchLinuxarm.org/platforms/armv6/raspberry-pi}{here},


\subsection{command lines}
\label{\detokenize{productmaint:command-lines}}
like this: \sphinxcode{\sphinxupquote{192.168.0.x}}, \sphinxcode{\sphinxupquote{10.0.0.14x}} or

Enough of networking for now. We’ll set a proper network configuration later in this guide, but first some \sphinxstyleemphasis{musthaves}.


\subsubsection{text block}
\label{\detokenize{productmaint:text-block}}
\fvset{hllines={, ,}}%
\begin{sphinxVerbatim}[commandchars=\\\{\}]
\PYG{n}{passwd}  \PYG{c+c1}{\PYGZsh{} change root password to something important}
\PYG{n}{rm} \PYG{o}{\PYGZhy{}}\PYG{n}{rf} \PYG{o}{/}\PYG{n}{etc}\PYG{o}{/}\PYG{n}{localtime}  \PYG{c+c1}{\PYGZsh{} dont care about this}
\PYG{n}{ln} \PYG{o}{\PYGZhy{}}\PYG{n}{s} \PYG{o}{/}\PYG{n}{usr}\PYG{o}{/}\PYG{n}{share}\PYG{o}{/}\PYG{n}{zoneinfo}\PYG{o}{/}\PYG{n}{Europe}\PYG{o}{/}\PYG{n}{Prague} \PYG{o}{/}\PYG{n}{etc}\PYG{o}{/}\PYG{n}{localtime}  \PYG{c+c1}{\PYGZsh{} set appropriate timezone}
\PYG{n}{echo} \PYG{l+s+s2}{\PYGZdq{}}\PYG{l+s+s2}{my\PYGZus{}raspberry}\PYG{l+s+s2}{\PYGZdq{}} \PYG{o}{\PYGZgt{}}  \PYG{o}{/}\PYG{n}{etc}\PYG{o}{/}\PYG{n}{hostname}  \PYG{c+c1}{\PYGZsh{} set name of your RPi}

\PYG{n}{useradd} \PYG{o}{\PYGZhy{}}\PYG{n}{m} \PYG{o}{\PYGZhy{}}\PYG{n}{aG} \PYG{n}{wheel} \PYG{o}{\PYGZhy{}}\PYG{n}{s} \PYG{o}{/}\PYG{n}{usr}\PYG{o}{/}\PYG{n+nb}{bin}\PYG{o}{/}\PYG{n}{bash} \PYG{n}{common\PYGZus{}user} \PYG{c+c1}{\PYGZsh{}}
\PYG{n}{groupadd} \PYG{n}{webdata}  \PYG{c+c1}{\PYGZsh{} for sharing}
\PYG{n}{useradd} \PYG{o}{\PYGZhy{}}\PYG{n}{M} \PYG{o}{\PYGZhy{}}\PYG{n}{aG} \PYG{n}{webdata} \PYG{o}{\PYGZhy{}}\PYG{n}{s} \PYG{o}{/}\PYG{n}{usr}\PYG{o}{/}\PYG{n+nb}{bin}\PYG{o}{/}\PYG{n}{false} \PYG{n}{nginx}
\PYG{n}{usermod} \PYG{o}{\PYGZhy{}}\PYG{n}{aG} \PYG{n}{webdata} \PYG{n}{common\PYGZus{}user}

\PYG{n}{visudo}  \PYG{c+c1}{\PYGZsh{} uncomment this line:  \PYGZpc{}wheel ALL=(ALL) ALL}

\PYG{n}{pacman} \PYG{o}{\PYGZhy{}}\PYG{n}{Syu}
\end{sphinxVerbatim}

\sphinxstylestrong{bold text}
\begin{itemize}
\item {} 
bullet

\item {} 
point

\end{itemize}


\section{Preferences}
\label{\detokenize{afterinstallconfig:preferences}}\label{\detokenize{afterinstallconfig::doc}}

\subsection{hyperlink}
\label{\detokenize{afterinstallconfig:hyperlink}}
Hyperlink \sphinxhref{http://ArchLinuxarm.org/platforms/armv6/raspberry-pi}{here},


\subsection{command lines}
\label{\detokenize{afterinstallconfig:command-lines}}
like this: \sphinxcode{\sphinxupquote{192.168.0.x}}, \sphinxcode{\sphinxupquote{10.0.0.14x}} or

Enough of networking for now. We’ll set a proper network configuration later in this guide, but first some \sphinxstyleemphasis{musthaves}.


\subsubsection{text block}
\label{\detokenize{afterinstallconfig:text-block}}
\fvset{hllines={, ,}}%
\begin{sphinxVerbatim}[commandchars=\\\{\}]
\PYG{n}{passwd}  \PYG{c+c1}{\PYGZsh{} change root password to something important}
\PYG{n}{rm} \PYG{o}{\PYGZhy{}}\PYG{n}{rf} \PYG{o}{/}\PYG{n}{etc}\PYG{o}{/}\PYG{n}{localtime}  \PYG{c+c1}{\PYGZsh{} dont care about this}
\PYG{n}{ln} \PYG{o}{\PYGZhy{}}\PYG{n}{s} \PYG{o}{/}\PYG{n}{usr}\PYG{o}{/}\PYG{n}{share}\PYG{o}{/}\PYG{n}{zoneinfo}\PYG{o}{/}\PYG{n}{Europe}\PYG{o}{/}\PYG{n}{Prague} \PYG{o}{/}\PYG{n}{etc}\PYG{o}{/}\PYG{n}{localtime}  \PYG{c+c1}{\PYGZsh{} set appropriate timezone}
\PYG{n}{echo} \PYG{l+s+s2}{\PYGZdq{}}\PYG{l+s+s2}{my\PYGZus{}raspberry}\PYG{l+s+s2}{\PYGZdq{}} \PYG{o}{\PYGZgt{}}  \PYG{o}{/}\PYG{n}{etc}\PYG{o}{/}\PYG{n}{hostname}  \PYG{c+c1}{\PYGZsh{} set name of your RPi}

\PYG{n}{useradd} \PYG{o}{\PYGZhy{}}\PYG{n}{m} \PYG{o}{\PYGZhy{}}\PYG{n}{aG} \PYG{n}{wheel} \PYG{o}{\PYGZhy{}}\PYG{n}{s} \PYG{o}{/}\PYG{n}{usr}\PYG{o}{/}\PYG{n+nb}{bin}\PYG{o}{/}\PYG{n}{bash} \PYG{n}{common\PYGZus{}user} \PYG{c+c1}{\PYGZsh{}}
\PYG{n}{groupadd} \PYG{n}{webdata}  \PYG{c+c1}{\PYGZsh{} for sharing}
\PYG{n}{useradd} \PYG{o}{\PYGZhy{}}\PYG{n}{M} \PYG{o}{\PYGZhy{}}\PYG{n}{aG} \PYG{n}{webdata} \PYG{o}{\PYGZhy{}}\PYG{n}{s} \PYG{o}{/}\PYG{n}{usr}\PYG{o}{/}\PYG{n+nb}{bin}\PYG{o}{/}\PYG{n}{false} \PYG{n}{nginx}
\PYG{n}{usermod} \PYG{o}{\PYGZhy{}}\PYG{n}{aG} \PYG{n}{webdata} \PYG{n}{common\PYGZus{}user}

\PYG{n}{visudo}  \PYG{c+c1}{\PYGZsh{} uncomment this line:  \PYGZpc{}wheel ALL=(ALL) ALL}

\PYG{n}{pacman} \PYG{o}{\PYGZhy{}}\PYG{n}{Syu}
\end{sphinxVerbatim}

\sphinxstylestrong{bold text}
\begin{itemize}
\item {} 
bullet

\item {} 
point

\end{itemize}


\section{Integration}
\label{\detokenize{integration:integration}}\label{\detokenize{integration::doc}}

\subsection{hyperlink}
\label{\detokenize{integration:hyperlink}}
Hyperlink \sphinxhref{http://ArchLinuxarm.org/platforms/armv6/raspberry-pi}{here},


\subsection{command lines}
\label{\detokenize{integration:command-lines}}
like this: \sphinxcode{\sphinxupquote{192.168.0.x}}, \sphinxcode{\sphinxupquote{10.0.0.14x}} or

Enough of networking for now. We’ll set a proper network configuration later in this guide, but first some \sphinxstyleemphasis{musthaves}.


\subsubsection{text block}
\label{\detokenize{integration:text-block}}
\fvset{hllines={, ,}}%
\begin{sphinxVerbatim}[commandchars=\\\{\}]
\PYG{n}{passwd}  \PYG{c+c1}{\PYGZsh{} change root password to something important}
\PYG{n}{rm} \PYG{o}{\PYGZhy{}}\PYG{n}{rf} \PYG{o}{/}\PYG{n}{etc}\PYG{o}{/}\PYG{n}{localtime}  \PYG{c+c1}{\PYGZsh{} dont care about this}
\PYG{n}{ln} \PYG{o}{\PYGZhy{}}\PYG{n}{s} \PYG{o}{/}\PYG{n}{usr}\PYG{o}{/}\PYG{n}{share}\PYG{o}{/}\PYG{n}{zoneinfo}\PYG{o}{/}\PYG{n}{Europe}\PYG{o}{/}\PYG{n}{Prague} \PYG{o}{/}\PYG{n}{etc}\PYG{o}{/}\PYG{n}{localtime}  \PYG{c+c1}{\PYGZsh{} set appropriate timezone}
\PYG{n}{echo} \PYG{l+s+s2}{\PYGZdq{}}\PYG{l+s+s2}{my\PYGZus{}raspberry}\PYG{l+s+s2}{\PYGZdq{}} \PYG{o}{\PYGZgt{}}  \PYG{o}{/}\PYG{n}{etc}\PYG{o}{/}\PYG{n}{hostname}  \PYG{c+c1}{\PYGZsh{} set name of your RPi}

\PYG{n}{useradd} \PYG{o}{\PYGZhy{}}\PYG{n}{m} \PYG{o}{\PYGZhy{}}\PYG{n}{aG} \PYG{n}{wheel} \PYG{o}{\PYGZhy{}}\PYG{n}{s} \PYG{o}{/}\PYG{n}{usr}\PYG{o}{/}\PYG{n+nb}{bin}\PYG{o}{/}\PYG{n}{bash} \PYG{n}{common\PYGZus{}user} \PYG{c+c1}{\PYGZsh{}}
\PYG{n}{groupadd} \PYG{n}{webdata}  \PYG{c+c1}{\PYGZsh{} for sharing}
\PYG{n}{useradd} \PYG{o}{\PYGZhy{}}\PYG{n}{M} \PYG{o}{\PYGZhy{}}\PYG{n}{aG} \PYG{n}{webdata} \PYG{o}{\PYGZhy{}}\PYG{n}{s} \PYG{o}{/}\PYG{n}{usr}\PYG{o}{/}\PYG{n+nb}{bin}\PYG{o}{/}\PYG{n}{false} \PYG{n}{nginx}
\PYG{n}{usermod} \PYG{o}{\PYGZhy{}}\PYG{n}{aG} \PYG{n}{webdata} \PYG{n}{common\PYGZus{}user}

\PYG{n}{visudo}  \PYG{c+c1}{\PYGZsh{} uncomment this line:  \PYGZpc{}wheel ALL=(ALL) ALL}

\PYG{n}{pacman} \PYG{o}{\PYGZhy{}}\PYG{n}{Syu}
\end{sphinxVerbatim}

\sphinxstylestrong{bold text}
\begin{itemize}
\item {} 
bullet

\item {} 
point

\end{itemize}


\section{Upgrading}
\label{\detokenize{upgrading:upgrading}}\label{\detokenize{upgrading::doc}}
The term ‘Upgrade’ is a term reserved for an alteration made to the software build itself, in which case an increase to the second but last digit is made to reflect this e.g. \sphinxcode{\sphinxupquote{0.X.0}}.  An Upgrade will more than likely require the user to re-download and install the software. The exception to this rule is in the case of major updates which can change the user experience so dramatically we mark the occaSiôn by changing the second but one digit of the version (as we do in the case of upgrades e.g. \sphinxcode{\sphinxupquote{0.X.0}})


\section{Troubleshooting}
\label{\detokenize{troubleshooting:troubleshooting}}\label{\detokenize{troubleshooting::doc}}

\subsection{hyperlink}
\label{\detokenize{troubleshooting:hyperlink}}
Hyperlink \sphinxhref{http://ArchLinuxarm.org/platforms/armv6/raspberry-pi}{here},


\subsection{command lines}
\label{\detokenize{troubleshooting:command-lines}}
like this: \sphinxcode{\sphinxupquote{192.168.0.x}}, \sphinxcode{\sphinxupquote{10.0.0.14x}} or

Enough of networking for now. We’ll set a proper network configuration later in this guide, but first some \sphinxstyleemphasis{musthaves}.


\subsubsection{text block}
\label{\detokenize{troubleshooting:text-block}}
\fvset{hllines={, ,}}%
\begin{sphinxVerbatim}[commandchars=\\\{\}]
\PYG{n}{passwd}  \PYG{c+c1}{\PYGZsh{} change root password to something important}
\PYG{n}{rm} \PYG{o}{\PYGZhy{}}\PYG{n}{rf} \PYG{o}{/}\PYG{n}{etc}\PYG{o}{/}\PYG{n}{localtime}  \PYG{c+c1}{\PYGZsh{} dont care about this}
\PYG{n}{ln} \PYG{o}{\PYGZhy{}}\PYG{n}{s} \PYG{o}{/}\PYG{n}{usr}\PYG{o}{/}\PYG{n}{share}\PYG{o}{/}\PYG{n}{zoneinfo}\PYG{o}{/}\PYG{n}{Europe}\PYG{o}{/}\PYG{n}{Prague} \PYG{o}{/}\PYG{n}{etc}\PYG{o}{/}\PYG{n}{localtime}  \PYG{c+c1}{\PYGZsh{} set appropriate timezone}
\PYG{n}{echo} \PYG{l+s+s2}{\PYGZdq{}}\PYG{l+s+s2}{my\PYGZus{}raspberry}\PYG{l+s+s2}{\PYGZdq{}} \PYG{o}{\PYGZgt{}}  \PYG{o}{/}\PYG{n}{etc}\PYG{o}{/}\PYG{n}{hostname}  \PYG{c+c1}{\PYGZsh{} set name of your RPi}

\PYG{n}{useradd} \PYG{o}{\PYGZhy{}}\PYG{n}{m} \PYG{o}{\PYGZhy{}}\PYG{n}{aG} \PYG{n}{wheel} \PYG{o}{\PYGZhy{}}\PYG{n}{s} \PYG{o}{/}\PYG{n}{usr}\PYG{o}{/}\PYG{n+nb}{bin}\PYG{o}{/}\PYG{n}{bash} \PYG{n}{common\PYGZus{}user} \PYG{c+c1}{\PYGZsh{}}
\PYG{n}{groupadd} \PYG{n}{webdata}  \PYG{c+c1}{\PYGZsh{} for sharing}
\PYG{n}{useradd} \PYG{o}{\PYGZhy{}}\PYG{n}{M} \PYG{o}{\PYGZhy{}}\PYG{n}{aG} \PYG{n}{webdata} \PYG{o}{\PYGZhy{}}\PYG{n}{s} \PYG{o}{/}\PYG{n}{usr}\PYG{o}{/}\PYG{n+nb}{bin}\PYG{o}{/}\PYG{n}{false} \PYG{n}{nginx}
\PYG{n}{usermod} \PYG{o}{\PYGZhy{}}\PYG{n}{aG} \PYG{n}{webdata} \PYG{n}{common\PYGZus{}user}

\PYG{n}{visudo}  \PYG{c+c1}{\PYGZsh{} uncomment this line:  \PYGZpc{}wheel ALL=(ALL) ALL}

\PYG{n}{pacman} \PYG{o}{\PYGZhy{}}\PYG{n}{Syu}
\end{sphinxVerbatim}

\sphinxstylestrong{bold text}
\begin{itemize}
\item {} 
bullet

\item {} 
point

\end{itemize}


\section{PDF’s \& Video’s}
\label{\detokenize{docsandvideos:pdf-s-video-s}}\label{\detokenize{docsandvideos::doc}}

\subsection{hyperlink}
\label{\detokenize{docsandvideos:hyperlink}}
Hyperlink \sphinxhref{http://ArchLinuxarm.org/platforms/armv6/raspberry-pi}{here},


\subsection{command lines}
\label{\detokenize{docsandvideos:command-lines}}
like this: \sphinxcode{\sphinxupquote{192.168.0.x}}, \sphinxcode{\sphinxupquote{10.0.0.14x}} or

Enough of networking for now. We’ll set a proper network configuration later in this guide, but first some \sphinxstyleemphasis{musthaves}.


\subsubsection{text block}
\label{\detokenize{docsandvideos:text-block}}
\fvset{hllines={, ,}}%
\begin{sphinxVerbatim}[commandchars=\\\{\}]
\PYG{n}{passwd}  \PYG{c+c1}{\PYGZsh{} change root password to something important}
\PYG{n}{rm} \PYG{o}{\PYGZhy{}}\PYG{n}{rf} \PYG{o}{/}\PYG{n}{etc}\PYG{o}{/}\PYG{n}{localtime}  \PYG{c+c1}{\PYGZsh{} dont care about this}
\PYG{n}{ln} \PYG{o}{\PYGZhy{}}\PYG{n}{s} \PYG{o}{/}\PYG{n}{usr}\PYG{o}{/}\PYG{n}{share}\PYG{o}{/}\PYG{n}{zoneinfo}\PYG{o}{/}\PYG{n}{Europe}\PYG{o}{/}\PYG{n}{Prague} \PYG{o}{/}\PYG{n}{etc}\PYG{o}{/}\PYG{n}{localtime}  \PYG{c+c1}{\PYGZsh{} set appropriate timezone}
\PYG{n}{echo} \PYG{l+s+s2}{\PYGZdq{}}\PYG{l+s+s2}{my\PYGZus{}raspberry}\PYG{l+s+s2}{\PYGZdq{}} \PYG{o}{\PYGZgt{}}  \PYG{o}{/}\PYG{n}{etc}\PYG{o}{/}\PYG{n}{hostname}  \PYG{c+c1}{\PYGZsh{} set name of your RPi}

\PYG{n}{useradd} \PYG{o}{\PYGZhy{}}\PYG{n}{m} \PYG{o}{\PYGZhy{}}\PYG{n}{aG} \PYG{n}{wheel} \PYG{o}{\PYGZhy{}}\PYG{n}{s} \PYG{o}{/}\PYG{n}{usr}\PYG{o}{/}\PYG{n+nb}{bin}\PYG{o}{/}\PYG{n}{bash} \PYG{n}{common\PYGZus{}user} \PYG{c+c1}{\PYGZsh{}}
\PYG{n}{groupadd} \PYG{n}{webdata}  \PYG{c+c1}{\PYGZsh{} for sharing}
\PYG{n}{useradd} \PYG{o}{\PYGZhy{}}\PYG{n}{M} \PYG{o}{\PYGZhy{}}\PYG{n}{aG} \PYG{n}{webdata} \PYG{o}{\PYGZhy{}}\PYG{n}{s} \PYG{o}{/}\PYG{n}{usr}\PYG{o}{/}\PYG{n+nb}{bin}\PYG{o}{/}\PYG{n}{false} \PYG{n}{nginx}
\PYG{n}{usermod} \PYG{o}{\PYGZhy{}}\PYG{n}{aG} \PYG{n}{webdata} \PYG{n}{common\PYGZus{}user}

\PYG{n}{visudo}  \PYG{c+c1}{\PYGZsh{} uncomment this line:  \PYGZpc{}wheel ALL=(ALL) ALL}

\PYG{n}{pacman} \PYG{o}{\PYGZhy{}}\PYG{n}{Syu}
\end{sphinxVerbatim}

\sphinxstylestrong{bold text}
\begin{itemize}
\item {} 
bullet

\item {} 
point

\end{itemize}


\section{FAQ’s and Other Resources}
\label{\detokenize{faq:faq-s-and-other-resources}}\label{\detokenize{faq::doc}}

\subsection{hyperlink}
\label{\detokenize{faq:hyperlink}}
Hyperlink \sphinxhref{http://ArchLinuxarm.org/platforms/armv6/raspberry-pi}{here},


\subsection{command lines}
\label{\detokenize{faq:command-lines}}
like this: \sphinxcode{\sphinxupquote{192.168.0.x}}, \sphinxcode{\sphinxupquote{10.0.0.14x}} or

Enough of networking for now. We’ll set a proper network configuration later in this guide, but first some \sphinxstyleemphasis{musthaves}.


\subsubsection{text block}
\label{\detokenize{faq:text-block}}
\fvset{hllines={, ,}}%
\begin{sphinxVerbatim}[commandchars=\\\{\}]
\PYG{n}{passwd}  \PYG{c+c1}{\PYGZsh{} change root password to something important}
\PYG{n}{rm} \PYG{o}{\PYGZhy{}}\PYG{n}{rf} \PYG{o}{/}\PYG{n}{etc}\PYG{o}{/}\PYG{n}{localtime}  \PYG{c+c1}{\PYGZsh{} dont care about this}
\PYG{n}{ln} \PYG{o}{\PYGZhy{}}\PYG{n}{s} \PYG{o}{/}\PYG{n}{usr}\PYG{o}{/}\PYG{n}{share}\PYG{o}{/}\PYG{n}{zoneinfo}\PYG{o}{/}\PYG{n}{Europe}\PYG{o}{/}\PYG{n}{Prague} \PYG{o}{/}\PYG{n}{etc}\PYG{o}{/}\PYG{n}{localtime}  \PYG{c+c1}{\PYGZsh{} set appropriate timezone}
\PYG{n}{echo} \PYG{l+s+s2}{\PYGZdq{}}\PYG{l+s+s2}{my\PYGZus{}raspberry}\PYG{l+s+s2}{\PYGZdq{}} \PYG{o}{\PYGZgt{}}  \PYG{o}{/}\PYG{n}{etc}\PYG{o}{/}\PYG{n}{hostname}  \PYG{c+c1}{\PYGZsh{} set name of your RPi}

\PYG{n}{useradd} \PYG{o}{\PYGZhy{}}\PYG{n}{m} \PYG{o}{\PYGZhy{}}\PYG{n}{aG} \PYG{n}{wheel} \PYG{o}{\PYGZhy{}}\PYG{n}{s} \PYG{o}{/}\PYG{n}{usr}\PYG{o}{/}\PYG{n+nb}{bin}\PYG{o}{/}\PYG{n}{bash} \PYG{n}{common\PYGZus{}user} \PYG{c+c1}{\PYGZsh{}}
\PYG{n}{groupadd} \PYG{n}{webdata}  \PYG{c+c1}{\PYGZsh{} for sharing}
\PYG{n}{useradd} \PYG{o}{\PYGZhy{}}\PYG{n}{M} \PYG{o}{\PYGZhy{}}\PYG{n}{aG} \PYG{n}{webdata} \PYG{o}{\PYGZhy{}}\PYG{n}{s} \PYG{o}{/}\PYG{n}{usr}\PYG{o}{/}\PYG{n+nb}{bin}\PYG{o}{/}\PYG{n}{false} \PYG{n}{nginx}
\PYG{n}{usermod} \PYG{o}{\PYGZhy{}}\PYG{n}{aG} \PYG{n}{webdata} \PYG{n}{common\PYGZus{}user}

\PYG{n}{visudo}  \PYG{c+c1}{\PYGZsh{} uncomment this line:  \PYGZpc{}wheel ALL=(ALL) ALL}

\PYG{n}{pacman} \PYG{o}{\PYGZhy{}}\PYG{n}{Syu}
\end{sphinxVerbatim}

\sphinxstylestrong{bold text}
\begin{itemize}
\item {} 
bullet

\item {} 
point

\end{itemize}


\section{Developers}
\label{\detokenize{developers:developers}}\label{\detokenize{developers::doc}}

\subsection{hyperlink}
\label{\detokenize{developers:hyperlink}}
Hyperlink \sphinxhref{http://ArchLinuxarm.org/platforms/armv6/raspberry-pi}{here},


\subsection{command lines}
\label{\detokenize{developers:command-lines}}
like this: \sphinxcode{\sphinxupquote{192.168.0.x}}, \sphinxcode{\sphinxupquote{10.0.0.14x}} or

Enough of networking for now. We’ll set a proper network configuration later in this guide, but first some \sphinxstyleemphasis{musthaves}.


\subsubsection{text block}
\label{\detokenize{developers:text-block}}
\fvset{hllines={, ,}}%
\begin{sphinxVerbatim}[commandchars=\\\{\}]
\PYG{n}{passwd}  \PYG{c+c1}{\PYGZsh{} change root password to something important}
\PYG{n}{rm} \PYG{o}{\PYGZhy{}}\PYG{n}{rf} \PYG{o}{/}\PYG{n}{etc}\PYG{o}{/}\PYG{n}{localtime}  \PYG{c+c1}{\PYGZsh{} dont care about this}
\PYG{n}{ln} \PYG{o}{\PYGZhy{}}\PYG{n}{s} \PYG{o}{/}\PYG{n}{usr}\PYG{o}{/}\PYG{n}{share}\PYG{o}{/}\PYG{n}{zoneinfo}\PYG{o}{/}\PYG{n}{Europe}\PYG{o}{/}\PYG{n}{Prague} \PYG{o}{/}\PYG{n}{etc}\PYG{o}{/}\PYG{n}{localtime}  \PYG{c+c1}{\PYGZsh{} set appropriate timezone}
\PYG{n}{echo} \PYG{l+s+s2}{\PYGZdq{}}\PYG{l+s+s2}{my\PYGZus{}raspberry}\PYG{l+s+s2}{\PYGZdq{}} \PYG{o}{\PYGZgt{}}  \PYG{o}{/}\PYG{n}{etc}\PYG{o}{/}\PYG{n}{hostname}  \PYG{c+c1}{\PYGZsh{} set name of your RPi}

\PYG{n}{useradd} \PYG{o}{\PYGZhy{}}\PYG{n}{m} \PYG{o}{\PYGZhy{}}\PYG{n}{aG} \PYG{n}{wheel} \PYG{o}{\PYGZhy{}}\PYG{n}{s} \PYG{o}{/}\PYG{n}{usr}\PYG{o}{/}\PYG{n+nb}{bin}\PYG{o}{/}\PYG{n}{bash} \PYG{n}{common\PYGZus{}user} \PYG{c+c1}{\PYGZsh{}}
\PYG{n}{groupadd} \PYG{n}{webdata}  \PYG{c+c1}{\PYGZsh{} for sharing}
\PYG{n}{useradd} \PYG{o}{\PYGZhy{}}\PYG{n}{M} \PYG{o}{\PYGZhy{}}\PYG{n}{aG} \PYG{n}{webdata} \PYG{o}{\PYGZhy{}}\PYG{n}{s} \PYG{o}{/}\PYG{n}{usr}\PYG{o}{/}\PYG{n+nb}{bin}\PYG{o}{/}\PYG{n}{false} \PYG{n}{nginx}
\PYG{n}{usermod} \PYG{o}{\PYGZhy{}}\PYG{n}{aG} \PYG{n}{webdata} \PYG{n}{common\PYGZus{}user}

\PYG{n}{visudo}  \PYG{c+c1}{\PYGZsh{} uncomment this line:  \PYGZpc{}wheel ALL=(ALL) ALL}

\PYG{n}{pacman} \PYG{o}{\PYGZhy{}}\PYG{n}{Syu}
\end{sphinxVerbatim}

\sphinxstylestrong{bold text}
\begin{itemize}
\item {} 
bullet

\item {} 
point

\end{itemize}


\section{Housekeeping}
\label{\detokenize{networkmaint:housekeeping}}\label{\detokenize{networkmaint::doc}}

\subsection{hyperlink}
\label{\detokenize{networkmaint:hyperlink}}
Hyperlink \sphinxhref{http://ArchLinuxarm.org/platforms/armv6/raspberry-pi}{here},


\subsection{command lines}
\label{\detokenize{networkmaint:command-lines}}
like this: \sphinxcode{\sphinxupquote{192.168.0.x}}, \sphinxcode{\sphinxupquote{10.0.0.14x}} or

Enough of networking for now. We’ll set a proper network configuration later in this guide, but first some \sphinxstyleemphasis{musthaves}.


\subsubsection{text block}
\label{\detokenize{networkmaint:text-block}}
\fvset{hllines={, ,}}%
\begin{sphinxVerbatim}[commandchars=\\\{\}]
\PYG{n}{passwd}  \PYG{c+c1}{\PYGZsh{} change root password to something important}
\PYG{n}{rm} \PYG{o}{\PYGZhy{}}\PYG{n}{rf} \PYG{o}{/}\PYG{n}{etc}\PYG{o}{/}\PYG{n}{localtime}  \PYG{c+c1}{\PYGZsh{} dont care about this}
\PYG{n}{ln} \PYG{o}{\PYGZhy{}}\PYG{n}{s} \PYG{o}{/}\PYG{n}{usr}\PYG{o}{/}\PYG{n}{share}\PYG{o}{/}\PYG{n}{zoneinfo}\PYG{o}{/}\PYG{n}{Europe}\PYG{o}{/}\PYG{n}{Prague} \PYG{o}{/}\PYG{n}{etc}\PYG{o}{/}\PYG{n}{localtime}  \PYG{c+c1}{\PYGZsh{} set appropriate timezone}
\PYG{n}{echo} \PYG{l+s+s2}{\PYGZdq{}}\PYG{l+s+s2}{my\PYGZus{}raspberry}\PYG{l+s+s2}{\PYGZdq{}} \PYG{o}{\PYGZgt{}}  \PYG{o}{/}\PYG{n}{etc}\PYG{o}{/}\PYG{n}{hostname}  \PYG{c+c1}{\PYGZsh{} set name of your RPi}

\PYG{n}{useradd} \PYG{o}{\PYGZhy{}}\PYG{n}{m} \PYG{o}{\PYGZhy{}}\PYG{n}{aG} \PYG{n}{wheel} \PYG{o}{\PYGZhy{}}\PYG{n}{s} \PYG{o}{/}\PYG{n}{usr}\PYG{o}{/}\PYG{n+nb}{bin}\PYG{o}{/}\PYG{n}{bash} \PYG{n}{common\PYGZus{}user} \PYG{c+c1}{\PYGZsh{}}
\PYG{n}{groupadd} \PYG{n}{webdata}  \PYG{c+c1}{\PYGZsh{} for sharing}
\PYG{n}{useradd} \PYG{o}{\PYGZhy{}}\PYG{n}{M} \PYG{o}{\PYGZhy{}}\PYG{n}{aG} \PYG{n}{webdata} \PYG{o}{\PYGZhy{}}\PYG{n}{s} \PYG{o}{/}\PYG{n}{usr}\PYG{o}{/}\PYG{n+nb}{bin}\PYG{o}{/}\PYG{n}{false} \PYG{n}{nginx}
\PYG{n}{usermod} \PYG{o}{\PYGZhy{}}\PYG{n}{aG} \PYG{n}{webdata} \PYG{n}{common\PYGZus{}user}

\PYG{n}{visudo}  \PYG{c+c1}{\PYGZsh{} uncomment this line:  \PYGZpc{}wheel ALL=(ALL) ALL}

\PYG{n}{pacman} \PYG{o}{\PYGZhy{}}\PYG{n}{Syu}
\end{sphinxVerbatim}

\sphinxstylestrong{bold text}
\begin{itemize}
\item {} 
bullet

\item {} 
point

\end{itemize}


\section{\sphinxstylestrong{Document Author(s):}}
\label{\detokenize{index:document-author-s}}

\subsection{Siôn H. Buckler}
\label{\detokenize{index:sion-h-buckler}}

\begin{savenotes}\sphinxattablestart
\centering
\begin{tabulary}{\linewidth}[t]{|T|T|T|}
\hline
\sphinxstyletheadfamily 
Organisation
&\sphinxstyletheadfamily 
Role
&\sphinxstyletheadfamily 
Details
\\
\hline
\noindent\sphinxincludegraphics{{wave-logo}.png}
&
Founder \& CEO
&
Make it Wave Ltd, British Corporation (England \& Wales), Company Director ID 11363386
\\
\hline
\noindent\sphinxincludegraphics{{ccu}.png}
&
Head of Defence
&
Caribbean Communications Unit (CCU), Royal Corps of Signals, Life Member ID 55983
\\
\hline
\noindent\sphinxincludegraphics{{uarsociety1}.png}
&
Council President
&
Utilities as a Right (UaaR) Society, British Public Servant, Gov/Oath ID 25148537
\\
\hline
\noindent\sphinxincludegraphics{{scottishbay}.png}
&
Military Theorist
&
Scottish Bay, Dominican Republic (Green Line \& Treaty of Guarantee)
\\
\hline
\end{tabulary}
\par
\sphinxattableend\end{savenotes}

\sphinxstylestrong{About} Siôn Buckler - Science \& Computer Science (Bachelors), Electronic Engineering, Industrial Electronics and Electronics \& Computing (Advanced Diplomas), Cisco Certified Network Associate (CCNA), Microsoft Certified Solutions Expert (MCSE), Certified Project Management (Prince2 Practitioner), Institute of Electronic Engineering (IEEE), Siemens Certified Engineer, Certified Telecommunications Service Provider (NVQ3), Satellites \& Full Spectrum Radio, Fixed Telecommunications Systems with Enhanced Capabilities , SKP01 Electrical Safety, NVQ2 IT, Defence Specialist LAN, TCP/IP, Subnetting, DHCP, Addressing, Routing \& Browsing, Communications Equitment Room Design \& Maintenance, Military Command \& Leaderership,  Cyber Security, Electronic Warfare, SIP/ VOIP, SEO, PPC, HTML5, CSS3, Java, Perl, Ajax, JQuery, MySQL, Unix, Python, Linux.



\renewcommand{\indexname}{Index}
\printindex
\end{document}