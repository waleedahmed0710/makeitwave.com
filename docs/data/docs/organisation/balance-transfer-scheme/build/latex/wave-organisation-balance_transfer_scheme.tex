%% Generated by Sphinx.
\def\sphinxdocclass{report}
\documentclass[letterpaper,10pt,openany,oneside,english]{sphinxmanual}
\ifdefined\pdfpxdimen
   \let\sphinxpxdimen\pdfpxdimen\else\newdimen\sphinxpxdimen
\fi \sphinxpxdimen=.75bp\relax

\PassOptionsToPackage{warn}{textcomp}
\usepackage[utf8]{inputenc}
\ifdefined\DeclareUnicodeCharacter
 \ifdefined\DeclareUnicodeCharacterAsOptional
  \DeclareUnicodeCharacter{"00A0}{\nobreakspace}
  \DeclareUnicodeCharacter{"2500}{\sphinxunichar{2500}}
  \DeclareUnicodeCharacter{"2502}{\sphinxunichar{2502}}
  \DeclareUnicodeCharacter{"2514}{\sphinxunichar{2514}}
  \DeclareUnicodeCharacter{"251C}{\sphinxunichar{251C}}
  \DeclareUnicodeCharacter{"2572}{\textbackslash}
 \else
  \DeclareUnicodeCharacter{00A0}{\nobreakspace}
  \DeclareUnicodeCharacter{2500}{\sphinxunichar{2500}}
  \DeclareUnicodeCharacter{2502}{\sphinxunichar{2502}}
  \DeclareUnicodeCharacter{2514}{\sphinxunichar{2514}}
  \DeclareUnicodeCharacter{251C}{\sphinxunichar{251C}}
  \DeclareUnicodeCharacter{2572}{\textbackslash}
 \fi
\fi
\usepackage{cmap}
\usepackage[T1]{fontenc}
\usepackage{amsmath,amssymb,amstext}
\usepackage[english]{babel}
\usepackage{times}
\usepackage[Bjarne]{fncychap}
\usepackage{sphinx}

\usepackage{geometry}

% Include hyperref last.
\usepackage{hyperref}
% Fix anchor placement for figures with captions.
\usepackage{hypcap}% it must be loaded after hyperref.
% Set up styles of URL: it should be placed after hyperref.
\urlstyle{same}

\addto\captionsenglish{\renewcommand{\figurename}{Fig.}}
\addto\captionsenglish{\renewcommand{\tablename}{Table}}
\addto\captionsenglish{\renewcommand{\literalblockname}{Listing}}

\addto\captionsenglish{\renewcommand{\literalblockcontinuedname}{continued from previous page}}
\addto\captionsenglish{\renewcommand{\literalblockcontinuesname}{continues on next page}}

\addto\extrasenglish{\def\pageautorefname{page}}

\setcounter{tocdepth}{1}



\title{Wave Credit Balance Transfer}
\date{Aug 15, 2019}
\release{0.0.3}
\author{Author(s): Make it Wave Ltd}
\newcommand{\sphinxlogo}{\vbox{}}
\renewcommand{\releasename}{Release}
\makeindex

\begin{document}

\maketitle
\sphinxtableofcontents
\phantomsection\label{\detokenize{index::doc}}


The goal of the Balance Transfer Scheme is to initiate the immidiate return of all investments and loans to Wave® financiers by September 2019.

Since Wave’s technology has existed in a state of research, development and trials for many years, funds from many investors and lenders have been tied into the project beyond the expected timescale. This scheme is the first of many to enable participants to access 100\% of their origional investment or loan (including any accumulated interest) on or before September 2019.

The banking application: Revolut, recently announced a cryptocurrency wallet feature, which works in conjunction with Wave® to replace the existing solution: Trustwallet. These Ethereum wallets are able to store Wave’s cryptocurrency Tokens and receive the corrsponding cryptocurrency royalties from Wave® - which instantly convert into almost any currency, transfer to almost any bank and/or withdraw from almost any ATM’s using the mobile application and Revolut MasterCard.

To celebrate Revolut’s aligned business interest with Wave®, the company is able and willing to release investors and lenders funds back to them. The amount is subject to some simple terms: The financier must create a Revolut account, deposit funds into it and present to Wave® for confirmation on or before the 18th of August 2019. Wave® will then release from their investment/loan account, 90\% of the funds they deposited to their new Revolut account - even if this amount equals 100\% of the loan/investment including any interest. This document will explain in detail the entirity of this scheme and how it works.


\chapter{Release Notes and Notices}
\label{\detokenize{releasenotes:release-notes-and-notices}}\label{\detokenize{releasenotes::doc}}
This section provides information about what is new or changed, including issues and documentation updates.
\begin{itemize}
\item {} 
‘Update’ is the term used to describe significant changes to our public source code. These technical documents is contained within our public source code.

\end{itemize}


\section{Version 0.0.3}
\label{\detokenize{releasenotes:version-0-0-3}}\begin{itemize}
\item {} \begin{enumerate}
\setcounter{enumi}{12}
\item {} 
Simpsons account queried, it’s showing nil tokens or investment.  Corrections made.

\end{enumerate}

\item {} \begin{enumerate}
\setcounter{enumi}{3}
\item {} 
Owens account queried, it’s showing nil investment.  Corrections made.

\end{enumerate}

\item {} \begin{enumerate}
\setcounter{enumi}{9}
\item {} 
Smith-Walker’s deposit showing £6k, when the deposit was supposedly £5k.  Corrections made.

\end{enumerate}

\item {} \begin{enumerate}
\setcounter{enumi}{9}
\item {} 
Davis reports the Revolut is asking for a premium upgrade or to share with 3 people to get the cryptocurrency wallet to work

\end{enumerate}

\end{itemize}


\section{Older Versions}
\label{\detokenize{releasenotes:older-versions}}
There are no older versions of this document at this time, however a table has been made so that enteries can be recorded:


\begin{savenotes}\sphinxattablestart
\centering
\sphinxcapstartof{table}
\sphinxcaption{Table 1.0 - Older Versions of this Document}\label{\detokenize{releasenotes:id1}}
\sphinxaftercaption
\begin{tabular}[t]{|\X{25}{100}|\X{25}{100}|\X{25}{100}|\X{25}{100}|}
\hline
\sphinxstyletheadfamily 
archive date
&\sphinxstyletheadfamily 
version
&\sphinxstyletheadfamily 
description
&\sphinxstyletheadfamily 
download link
\\
\hline
14-08-2019
&
0.0.2
&
see notes
&
\sphinxhref{https://makeitwave.com/data/docs/organisation/balance-transfer-scheme/build/html/\_static/archived/version-0.0.2.pdf}{version-0.0.2.pdf}
\\
\hline
13-08-2019
&
0.0.1
&
1st Draft
&
\sphinxhref{https://makeitwave.com/data/docs/organisation/balance-transfer-scheme/build/html/\_static/archived/version-0.0.1.pdf}{version-0.0.1.pdf}
\\
\hline
\end{tabular}
\par
\sphinxattableend\end{savenotes}


\subsection{Version 0.0.2}
\label{\detokenize{releasenotes:version-0-0-2}}\begin{itemize}
\item {} 
Added the account balances to the document

\item {} 
Removed ability to withdraw gratuity on Royalty Tokens. Initial investment + interest only!

\item {} 
Closing date for submitting Revolut balances to Wave has changed from the 16th to the 18th August 2019

\end{itemize}


\subsection{Version 0.0.1}
\label{\detokenize{releasenotes:version-0-0-1}}\begin{itemize}
\item {} 
This is the first release/ draft of this document.

\item {} 
The content of the document can be navigated from the index.

\end{itemize}


\subsection{Version 0.0.0}
\label{\detokenize{releasenotes:version-0-0-0}}
N/A


\section{Known and Corrected Issues}
\label{\detokenize{releasenotes:known-and-corrected-issues}}
Below is a table of pending issues which have been reported to our team. These issues will be cleared from this list as and when they are remedied.


\begin{savenotes}\sphinxattablestart
\centering
\sphinxcapstartof{table}
\sphinxcaption{Table 1.1 - Known Issues}\label{\detokenize{releasenotes:id2}}
\sphinxaftercaption
\begin{tabular}[t]{|\X{10}{100}|\X{10}{100}|\X{20}{100}|\X{60}{100}|}
\hline
\sphinxstyletheadfamily 
date
&\sphinxstyletheadfamily 
version
&\sphinxstyletheadfamily 
subject
&\sphinxstyletheadfamily 
description
\\
\hline
01-08-2019
&
0.0.1
&
N/A
&
no issues to report - first draft only
\\
\hline
\end{tabular}
\par
\sphinxattableend\end{savenotes}

\sphinxstylestrong{Comments} - none


\chapter{Introduction}
\label{\detokenize{introduction:introduction}}\label{\detokenize{introduction::doc}}
Wave® is a free internet and iptv service, available in the form of a free software upgrade for supported LTE, 5G and Satcom (Starlink) Wireless Receiver/ Hotspots. Wave® makes internet and iptv free by mining cryptocurrency and generating advertising revenue from every browser-enabled mobile device connected to these Smart Home Hotspots. Built into the operating system and applications are cryptocurrency smart contracts which autonomously distribute this revenue to their respective geographical regions’ development teams and Internet/ IPTV Service Providers - resulting in a subscription-free internet and iptv service for the end-user.

\noindent\sphinxincludegraphics{{wave-screenshot}.png}

This same method is also used to remunerate Wave’s investors and lenders. These types of financiers are represented with a corresponding class (and supply) of a Ethereum-based cryptocurrency tokens. Terms such as ‘shares certificates’ and ‘Promissory Notes’ are subsequently replaced with Wave® Royalty Tokens(WRT) and Wave® Interest Tokens(WIT).

Tokenising the various stakes in Wave® in this way, permits them to be freely exchanged between the cryptocurrency wallet addresses of every ‘Token Holder’. The entire history of where these tokens are stored (at any given point in time) is also public record, which is how Wave® is able to autonomously distribute revenue to the same wallet address as the Token Holders. The result is free, paperless, secure, de-centralized and annonomous trading of Wave® Tokens in addition to the cryptocurrency they generate. The Wave® website and software itself is also public, permitting developers to be remunerated in a similar way. This autonomous business is known as a Decentraliased Autonomous Organisaion (DAO).

As the software develops and the network scales, the revenue distributed to Token Holders (and subsequent value of the tokens) will grow,  creating a better incentive and more contribution and involvement from Token Holders, which only further accelerates the growth. This autonomous revenue distribution solution works in two parts. Built-into into the Wave® operating system is technology which generates and distributes the revenue. Now Wave® is introducing the final part of this model: Revolut! a new mobile banking application described below, for Token Holders to install to their mobile devices. Once stakeholders have generated and shared their cryptocurrency wallets with Revolut, the corresponding Tokens can be issued. The distribution of these tokens is known as an Initial Token Offering or Initial Coin Offering (ICO).


\section{Revolut}
\label{\detokenize{introduction:revolut}}
Revolut Ltd is a “UK financial technology company that offers banking services including a pre-paid debit card (Mastercard or Visa), currency exchange, cryptocurrency exchange and peer-to-peer payments. The Revolut mobile app supports spending and ATM withdrawals in 120 currencies and sending in 29 currencies directly from the app.

\noindent\sphinxincludegraphics{{revolt}.png}

Revolut also provides customers access to cryptocurrencies by exchanging to or from 25 fiat currencies. Revolut currently charges no fees for the majority of its services (but for a capped usage), and uses interbank exchange rates for its currency exchange”. Some of the many benefits of using the Revolut Mobile Banking Application include:
\begin{itemize}
\item {} 
support for storing and transfering Government-issued and cryptographic currencies

\item {} 
a securities exchange for instant transfers between central-bank and crypto currencies

\item {} 
a virtual and physical MasterCards for online purchases and cashpoint money withdrawls

\end{itemize}

Since Revolut unlocked its cryptocurrency feature in July 2019, Wave® Token Holders are now able to use the App to store their Wave® Tokens and receive the corresponding cryptocurrency royalties which come from owning the Token. These royalties can then be converted into Pound Sterling and/or US Dollars (and a range of other currencies) from within the same application … with the added advantage of instant (local or international) bank-to-bank(BACS) transfers, additional currency exchanges and standing orders/ direct debits etc.


\chapter{How it Works}
\label{\detokenize{howitworks:how-it-works}}\label{\detokenize{howitworks::doc}}
Wave® and its financiers both have various forms of credit facilities avaliable. This scheme uses these credit facilities in conjunction with each other to return investments and loans to Wave® financiers by September 2019.

In the case of Wave®, the company has excellent credit with its card merchants e.g. Stripe, Paypal etc. Following the rare circumstance where a card payment is taken and a refund is made, a situation is created where both the card holder and Wave® are simultaniously in receipt of the payment. This is because the card merchant uses their own capital to process the refund then recouporates the amount later by deducting it from future card payments before they reach Wave®.

In the case of Wave’s financiers, credit is avaliable to them in various other forms e.g. loans, credit cards etc. Traditionally this opportunity is not fully explored due to high-rates of interest or limited requirement for borrowing due to alternative funding sources e.g. savings, salary etc. But transfering these lines of credit from a high interest lender to lower interest lenders is a sound act and known as a balance transfer. The goal for the financier will be to get their Revolut account balance to exceed their investment/ loan to Wave® before the closing submission date of this scheme. Once Wave® has confirmed the balance with a Credit/Debit transaction, 90\% of the amount confirmed will be taken from the Wave® Investment/ Loan account of the financier and released to their Revolut account, all before the 1st September 2019.

The benefit to Wave® is that it simultaniously recouperates some equity and liberates itself from the compounding high rates of interest it currently provides its investors and lenders. However the funds being returned to the financiers is not from Wave® releasing working capital, instead Wave® is systematically transfering its obligations from seed fund financiers to its card merchants. As of September 2019 Wave® will (in full or in part) be indebted to its card merchant(s) and not its origional financier. The liability to the card merchants is 0\% APR as oppose to the 10 - 22.5\% APR currently compounding on the accounts of private investors/ lenders.


\section{Example}
\label{\detokenize{howitworks:example}}
In the example below a financier has loaned Wave® £10,000 GBP. The financier also has access to a further £10,000. In addition Wave® has access to £10,000 from its own 3rd party lender, in this case its card merchant.  Wave® withdraws and refunds the financiers avaliable balance, which can be credit but not a requirement, creating a situation where Wave’s card merchant willingly incurs liabiliy while Wave® is in a position of positive credit. Less the card transaction fees Wave® then transfers this positive credit to the finacier. Wave® is then liable to Stripe and not the financier.


\begin{savenotes}\sphinxattablestart
\centering
\sphinxcapstartof{table}
\sphinxcaption{Table 1.2 - Example Balance Transfer}\label{\detokenize{howitworks:id1}}
\sphinxaftercaption
\begin{tabular}[t]{|\X{25}{100}|\X{25}{100}|\X{25}{100}|\X{25}{100}|}
\hline
\sphinxstyletheadfamily 
Details
&\sphinxstyletheadfamily 
Financier
&\sphinxstyletheadfamily 
Merchant
&\sphinxstyletheadfamily 
Wave
\\
\hline
Investment/ Loan Tx
&
-10000
&
0
&
10000
\\
\hline
Balance Check Tx
&
-20000
&
0
&
20000
\\
\hline
Balance Check Rx
&
-10000
&
-10000
&
20000
\\
\hline
CBT Payout Rx
&
-1000
&
-10000
&
11000
\\
\hline&&&\\
\hline&&&\\
\hline
\end{tabular}
\par
\sphinxattableend\end{savenotes}


\chapter{Proof of Concept}
\label{\detokenize{proofofconcept:proof-of-concept}}\label{\detokenize{proofofconcept::doc}}
In the last two years Wave® has issued a total of only four refunds of any purposeful amounts. In all these instances the funds were returned to the cardholder using the capital of the card merchant and not Wave®. These transactions prove the theory behind this scheme, not withstanding the timescales. The balance confirmation detailed in Step 3 of the ‘Getting Started’ section, will perform a debit and credit (refund) much faster than the timescales of the examples below. The expectation is between 5 and 10 days, providing Step 3 is completed before the closing date of this scheme:

\sphinxstylestrong{Eddy Van-Tricht} paid \$1,050.00 USD to Wave® on the 13th March 2018 and was refunded on the 11th May 2018 using the card merchant Stripe.

\sphinxstylestrong{Johnny Choudhury-Lucas} paid £1,483.13 GBP to Wave® on the 3rd April 2018 and was refunded on the 29th April 2018 using the card merchant Stripe.

\sphinxstylestrong{Leyda Roa} paid \$400 USD to Wave® on the 3rd September 2018 and was refunded by the 27th September 2018 using the card merchant Paypal.

\sphinxstylestrong{Elizabeth Coldwell-Hall} paid £400 GBP to Wave® on the 23rd July 2019 and was refunded on the 9th August 2019 using the card merchant Paypal.


\chapter{Account Balances}
\label{\detokenize{accounts:account-balances}}\label{\detokenize{accounts::doc}}
Below is a list of the account balances for the various types of Wave® cryptocurrency Tokens.

In order to calculate and release your full investment and loan in Wave (including all interest), divide your account balance by 0.9. This is the amount you will be required to deposit to your new Revolut account.


\section{WIT Account(s)}
\label{\detokenize{accounts:wit-account-s}}
In the case of WIT Accounts, 10\% APR has been compounding since the date funds are deposited, this was introduced and backdated for participating financiers in December 2017.


\begin{savenotes}\sphinxattablestart
\centering
\sphinxcapstartof{table}
\sphinxcaption{Table 1.3 - Wave® Interest Token (WIT) Summary}\label{\detokenize{accounts:id1}}
\sphinxaftercaption
\begin{tabular}[t]{|\X{10}{100}|\X{25}{100}|\X{20}{100}|\X{20}{100}|\X{25}{100}|}
\hline
\sphinxstyletheadfamily 
UID
&\sphinxstyletheadfamily 
Wallet ID/ Alias
&\sphinxstyletheadfamily 
Deposited (GBP)
&\sphinxstyletheadfamily 
Interest 10\% (GBP)
&\sphinxstyletheadfamily 
Balance (GBP)
\\
\hline
201
&
J-Rushton
&
18218.80
&
12021.45
&
30240.25
\\
\hline
202
&
M-Glover
&
25190.00
&
22069.79
&
47259.79
\\
\hline
203
&
S-Murphy
&
5190.00
&
3648.50
&
8838.50
\\
\hline
204
&
Family-Gill
&
40190.00
&
33575.19
&
73575.19
\\
\hline
205
&
D-Khan
&
2711.49
&
971.98
&
3683.47
\\
\hline
206
&
M-Kennedy
&
5055.13
&
4307.30
&
8586.32
\\
\hline
207
&
R-Davis
&
3000.00
&
657.13
&
3657.13
\\
\hline
208
&
F-Van-Rienen
&
6000.00
&
1581.63
&
7581.63
\\
\hline
209
&
K-E-Amos
&
1680.00
&
168.60
&
1848.60
\\
\hline
210
&
Anonymous-1
&
1103.00
&
87.35
&
1190.35
\\
\hline&&&&\\
\hline&
Total
&
108338.42
&
79088.91
&
186461.22
\\
\hline
\end{tabular}
\par
\sphinxattableend\end{savenotes}


\section{WRT Account(s)}
\label{\detokenize{accounts:wrt-account-s}}
In the case of WRT Accounts, 16.6\% APR began being added to the sum deposited from April 2019.


\begin{savenotes}\sphinxatlongtablestart\begin{longtable}{|\X{10}{100}|\X{30}{100}|\X{20}{100}|\X{15}{100}|\X{25}{100}|}
\caption{Table 1.4 - Wave® Royalty Tokens (WRT) - Summary\strut}\label{\detokenize{accounts:id2}}\\*[\sphinxlongtablecapskipadjust]
\hline
\sphinxstyletheadfamily 
UID
&\sphinxstyletheadfamily 
Wallet ID/ Alias
&\sphinxstyletheadfamily 
Deposited (GBP)
&\sphinxstyletheadfamily 
Interest (GBP)
&\sphinxstyletheadfamily 
Balance (GBP)
\\
\hline
\endfirsthead

\multicolumn{5}{c}%
{\makebox[0pt]{\sphinxtablecontinued{\tablename\ \thetable{} -- continued from previous page}}}\\
\hline
\sphinxstyletheadfamily 
UID
&\sphinxstyletheadfamily 
Wallet ID/ Alias
&\sphinxstyletheadfamily 
Deposited (GBP)
&\sphinxstyletheadfamily 
Interest (GBP)
&\sphinxstyletheadfamily 
Balance (GBP)
\\
\hline
\endhead

\hline
\multicolumn{5}{r}{\makebox[0pt][r]{\sphinxtablecontinued{Continued on next page}}}\\
\endfoot

\endlastfoot

101
&
Seed-Round-B-Leonides
&
0
&
0.00
&
0.00
\\
\hline
102
&
Seed-Round-M-Holmes
&
500
&
30.93
&
530.93
\\
\hline
103
&
Seed-Round-R-James
&
500
&
30.93
&
530.93
\\
\hline
104
&
Seed-Round-S-Barnes
&
500
&
30.93
&
530.93
\\
\hline
105
&
Seed-Round-n/a
&
0
&
0.00
&
0.00
\\
\hline
106
&
Seed-Round-P-Beals
&
2000
&
123.70
&
2123.70
\\
\hline
107
&
Seed-Round-P-Brown
&
0
&
0.00
&
0.00
\\
\hline
108
&
Seed-Round-M-Hough
&
0
&
0.00
&
0.00
\\
\hline
109
&
Seed-Round-A-Witcomb
&
0
&
0.00
&
0.00
\\
\hline
110
&
Seed-Round-D-Owen
&
5000
&
309.26
&
5309.26
\\
\hline
111
&
Seed-Round-J-Smith-Walker
&
5000
&
309.26
&
5309.26
\\
\hline
112
&
Seed-Round-S-Gates
&
0
&
0.00
&
0.00
\\
\hline
113
&
Seed-Round-C-A-Doick
&
2000
&
123.70
&
2123.70
\\
\hline
114
&
Seed-Round-C-Kell
&
4000
&
247.41
&
4247.41
\\
\hline
115
&
Seed-Round-S-Purcell
&
4500
&
278.33
&
4778.33
\\
\hline
116
&
Seed-Round-C-Chapman
&
1000
&
61.85
&
1061.85
\\
\hline
117
&
Seed-Round-T-Marshall
&
1000
&
61.85
&
1061.85
\\
\hline
118
&
Seed-Round-C-Marshall
&
3000
&
185.56
&
3185.56
\\
\hline
119
&
Seed-Round-H-Davies
&
0
&
0.00
&
0.00
\\
\hline
120
&
Seed-Round-M-Simpson
&
900
&
55.67
&
955.67
\\
\hline
121
&
Seed-Round-S-Hume
&
1000
&
61.85
&
1061.85
\\
\hline
122
&
Seed-Round-B-Pullen
&
8000
&
494.82
&
8494.82
\\
\hline
123
&
Seed-Round-G-Caines
&
300
&
18.56
&
318.56
\\
\hline
124
&
Seed-Round-A-Powell
&
3000
&
185.56
&
3185.56
\\
\hline
125
&
Seed-Round-M-Preston
&
0
&
0.00
&
0.00
\\
\hline
126
&
Seed-Round-S-Chapman
&
100
&
6.19
&
106.19
\\
\hline
127
&
Seed-Round-L-Wallace
&
0
&
0.00
&
0.00
\\
\hline
128
&
Seed-Round-M-Boyd
&
2000
&
123.70
&
2123.70
\\
\hline
129
&
Seed-Round-M-Gerard
&
5000
&
309.26
&
5309.26
\\
\hline
130
&
Seed-Round-S-Hargreaves
&
0
&
0.00
&
0.00
\\
\hline
131
&
Seed-Round-G-Stewart
&
0
&
0.00
&
0.00
\\
\hline
132
&
Seed-Round-S-Reynolds
&
0
&
0.00
&
0.00
\\
\hline
133
&
Seed-Round-D-Allen
&
0
&
0.00
&
0.00
\\
\hline
134
&
Seed-Round-J-Davis
&
7700
&
476.26
&
8176.26
\\
\hline
135
&
Seed-Round-N-Smith
&
0
&
0.00
&
0.00
\\
\hline
136
&
Seed-Round-D-Marshall
&
15000
&
927.78
&
15927.78
\\
\hline
137
&
Seed-Round-A-Vashi
&
1000
&
61.85
&
1061.85
\\
\hline
138
&
Seed-Round-C-Pitcairn
&
0
&
0.00
&
0.00
\\
\hline
139
&
Seed-Round-H-Pitcairn
&
500
&
30.93
&
530.93
\\
\hline
140
&
Seed-Round-L-Allen
&
74000
&
4577.05
&
78577.05
\\
\hline
141
&
Seed-Round-G-Pitcairn
&
0
&
0.00
&
0.00
\\
\hline
142
&
Seed-Round-P-Caines
&
41354.2
&
2557.84
&
43912.04
\\
\hline
143
&
Seed-Round-S-Buckler
&
67000
&
4144.09
&
71144.09
\\
\hline
144
&
Seed-Round-E-Young
&
0
&
0.00
&
0.00
\\
\hline
145
&
Seed-Round-J-O-Sullivan
&
0
&
0.00
&
0.00
\\
\hline
146
&
Seed-Round-Y-Sakowitz
&
0
&
0.00
&
0.00
\\
\hline
147
&
Seed-Round-M-Weaver
&
0
&
0.00
&
0.00
\\
\hline
148
&
Seed-Round-J-Choudhury-Lucas
&
3268.75
&
202.18
&
3470.93
\\
\hline
149
&
Seed-Round-R-Stevenson
&
0
&
0.00
&
0.00
\\
\hline
150
&
Seed-Round-B-Naipaul
&
0
&
0.00
&
0.00
\\
\hline&&&&\\
\hline&&&&\\
\hline&

&&&\\
\hline&
Total (GBP)
&
259122.95
&
16027.29
&
275150.24
\\
\hline
\end{longtable}\sphinxatlongtableend\end{savenotes}


\section{WIN Account(s)}
\label{\detokenize{accounts:win-account-s}}
In the case of WIT Token Holders, a rate of 22.5\% APR is added to the account, subject to a minimum monthly deposit of £50 GBP.


\begin{savenotes}\sphinxattablestart
\centering
\sphinxcapstartof{table}
\sphinxcaption{Table 1.5 - Wave® Impact Notes (WIN) - Summary}\label{\detokenize{accounts:id3}}
\sphinxaftercaption
\begin{tabular}[t]{|\X{20}{100}|\X{20}{100}|\X{20}{100}|\X{20}{100}|\X{20}{100}|}
\hline
\sphinxstyletheadfamily 
UID
&\sphinxstyletheadfamily 
Wallet ID/ Alias
&\sphinxstyletheadfamily 
Deposited (GBP)
&\sphinxstyletheadfamily 
Interest (GBP)
&\sphinxstyletheadfamily 
Balance (GBP)
\\
\hline
300
&
Coldwell
&
3350.00
&
221.79
&
3571.79
\\
\hline
301
&&&&\\
\hline
\end{tabular}
\par
\sphinxattableend\end{savenotes}


\chapter{Getting Started}
\label{\detokenize{getstarted:getting-started}}\label{\detokenize{getstarted::doc}}

\section{Step 1 - Revolut Account Creation}
\label{\detokenize{getstarted:step-1-revolut-account-creation}}
To setup your Revolut Bank visit \sphinxurl{https://www.revolut.com/} and download the application to your mobile device.
Revolut will require you to take a photo of your passport and upload a selfi photo to confirm your identity.


\section{Step 2 - Top-Up \& Virtual Card}
\label{\detokenize{getstarted:step-2-top-up-virtual-card}}
The Revolut banking app will provide you with a sort code and account number so that you can deposit funds to your new account.
You may also money transfer funds from credit cards. The goal of this scheme it to get your Revolut account balance as high as possible before submitting the balance to Wave® on or before the 18th August 2019. The more the balance, the more funds you will be able to release from your investment/ loan account. Before proceeding to step 3 you will need to generate a virtual bank card.


\section{Step 3 - Balance Confirmation}
\label{\detokenize{getstarted:step-3-balance-confirmation}}
Next Wave® will confirm your Revolut account balance.
Simply visit the link below and follow the onscreen steps: \sphinxurl{https://donorbox.org/check-balance}

Your Revolut virtual bank card will be debited and credited to confirm the Revolut account balance.
The form requires your main bank account details (preferably Revolut) so as the investment/loan can be deposited.


\section{Step 4 - Processing \& Completion}
\label{\detokenize{getstarted:step-4-processing-completion}}
The deadline for Step 3 is the 18th August 2019. The card merchant is Stripe, who take between 5 and 10 days to process the transaction. And a further 5-10 days to refund the transaction back to the card holder. In this time the financiers investment/ loan (equal to 90\% of the confirmed Revolut account balance) will be deposited to their bank account submitted in the form in Step 3. All records will be re-published to the document section of our website in Septmeber 2019 to reflect the activity of this scheme.

While waiting for these funds to process, please consider ordering a physical bank card and creating a virtual currency wallet (Ethereum) within your Revolut banking application.


\chapter{Terms \& Conditions}
\label{\detokenize{terms:terms-conditions}}\label{\detokenize{terms::doc}}
See below the Terms and Conditions of the Wave® Credit Balance Transfer Scheme:
\begin{itemize}
\item {} 
The funds released from the Wave® investment/ loan account will equal, but not exceed 90\% of the reported peak balance of the financiers Revolut account, on or before the 18th August 2019

\item {} 
The maximum amount to be released from Wave® cannot exceed the financiers investment/loan account with Wave®

\item {} 
Compounded interest and gratuity on all loans and investments is included in this scheme

\item {} 
The financier may deposit to Revolut, less than their Wave® account balance and withdraw less than their origional full investment/ loan

\item {} 
The financiers Revolut account balance must be declaired to and verified by Wave® before the 18th August 2019

\item {} 
Wave® will Debit/Credit the Revolut account to confirm the balance and then release from the corresponding investment/ loan account 90\% of said balance, before the 1st September 2019

\item {} 
The virtual card created in the Revolut mobile app is sufficient for this scheme - a physical bank card is optional, but advised for future benefits

\end{itemize}


\chapter{\sphinxstylestrong{Document Author(s):}}
\label{\detokenize{index:document-author-s}}

\section{Siôn H. Buckler}
\label{\detokenize{index:sion-h-buckler}}

\begin{savenotes}\sphinxattablestart
\centering
\begin{tabulary}{\linewidth}[t]{|T|T|T|}
\hline
\sphinxstyletheadfamily 
Organisation
&\sphinxstyletheadfamily 
Role
&\sphinxstyletheadfamily 
Details
\\
\hline
\noindent\sphinxincludegraphics{{wave-logo}.png}
&
Founder \& CEO
&
Make it Wave Ltd, British Corporation (England \& Wales), Company Director ID 11363386
\\
\hline
\noindent\sphinxincludegraphics{{ccu}.png}
&
Head of Defence
&
Caribbean Communications Unit (CCU), Royal Corps of Signals, Life Member ID 55983
\\
\hline
\noindent\sphinxincludegraphics{{uarsociety1}.png}
&
Technical Author
&
X as a Right (XaaR) e.g. Internet as a Right, British Goverment/Oath ID 25148537
\\
\hline
\noindent\sphinxincludegraphics{{scottishbay}.png}
&
Military Theorist
&
Scottish Bay Society (Green Line \& Treaty of Guarantee), Dominican Republic
\\
\hline
\end{tabulary}
\par
\sphinxattableend\end{savenotes}

\sphinxstylestrong{About: Siôn Buckler} - Science \& Computer Science (Bachelors), Electronic Engineering, Industrial Electronics and Electronics \& Computing (Advanced Diplomas), Cisco Certified Network Associate (CCNA), Microsoft Certified Solutions Expert (MCSE), Certified Project Management (Prince2 Practitioner), Institute of Electronic Engineering (IEEE), Siemens Certified Engineer, Certified Telecommunications Service Provider (NVQ3), Satellites \& Full Spectrum Radio, Fixed Telecommunications Systems with Enhanced Capabilities , SKP01 Electrical Safety, NVQ2 IT, Defence Specialist LAN, TCP/IP, Subnetting, DHCP, Addressing, Routing \& Browsing, Communications Equitment Room Design \& Maintenance, Military Command \& Leaderership,  Cyber Security, Electronic Warfare, SIP/ VOIP, SEO, PPC, HTML5, CSS3, Java, Perl, Ajax, JQuery, MySQL, Unix, Python, Linux.



\renewcommand{\indexname}{Index}
\printindex
\end{document}